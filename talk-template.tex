\documentclass[smaller,aspectratio=149]{beamer}
% \usepackage[utf8]{inputenc}

\title{Concise descriptive title with important keywords}
% \subtitle{Lighthearted subtitle, perhaps}
\author{\textbf{First Author}, Second author, Last author}
\institute{Some University, Perhaps with location}
\date{Conference Name 2020}

% Formatting
\usetheme{matryoshkev}
\graphicspath{ {./images/} } % Folder where images are stored

\begin{document}

%% 
%% TITLE SLIDE
%% 

\frame{\titlepage}

% Acknowledgments
\begin{frame}{}
	\begin{centering}
		\begin{minipage}[t]{0.3\linewidth}
			\begin{centering}
				\frame{\includegraphics[height=0.35\textheight,width=0.3\textheight]{example-image}}\par
				\textbf{Second Author}\\{\footnotesize Some University}\par
			\end{centering}
		\end{minipage}
		\begin{minipage}[t]{0.3\linewidth}
			\begin{centering}
				\frame{\includegraphics[height=0.35\textheight,width=0.28\textheight]{example-image}}\par
				\textbf{Last Author}\\{\footnotesize University of Somewhere}\par
			\end{centering}
		\end{minipage}
		\begin{minipage}[t]{0.35\linewidth}
			\begin{centering}
				\frame{\includegraphics[height=0.35\textheight]{example-image}}\par
				\textbf{Undergrad 1}\\\textbf{Undergrad 2}\\{\footnotesize My Institute}\par
			\end{centering}
		\end{minipage}
	\end{centering}\par
	\bigskip\medskip
	\begin{small}
		Don't flub the ending---make your acknowledgements here \par
		\smallskip Include funding you're responsible for \par
	\end{small}
\end{frame}

%% Plain text slide
%\begin{frame}{Slide title text}
%% \framesubtitle{Subtitle for context?}
%	Plain text in a slide with no environment. Generally it's not a good idea to have large blocks of text on a slide. Especially with no image. If the audience is reading, they're not listening to you.\par 
%	\medskip Instead try to:
%	\begin{itemize}\setlength\itemsep{0pt}
%		\item Use short sentence fragments
%		\item Include an image
%		\item Make slide title the main message
%	\end{itemize}
%	\smallskip Paragraphs spaced manually with \texttt{smallskip}, \texttt{medskip}, \textit{etc.}\par 
%	\bigskip \textbf{Bold text}\par
%	{\small Lets you make definitions}\par
%	\medskip \textbf{Bold text}\par
%	{\small Also works for comparisons}\par
%\end{frame}

%% 
%% INTRODUCTION
%% 

% Two columns, right image
\begin{frame}{}
	\begin{columns}[T]
		\column{0.62\textwidth}
		\titlemessage{Use titles to state slide's main message}
		\bigskip\bigskip
		Include image illustrating main message\par
		\bigskip Just enough text to catch up if someone didn't hear you
		\begin{itemize}\setlength\itemsep{0pt}
			\item Use short sentence fragments
			\item If audience is reading, they're not listening
		\end{itemize}
		\column{0.38\textwidth}
		\begin{centering}%
			\frame{\includegraphics[width=\linewidth,height=0.75\paperheight]{example-image}}\par%
		\raggedleft\tiny{Image credit}\par
		\end{centering}	
	\end{columns}
\end{frame}

% Vary layout
\begin{frame}{}
\begin{columns}[T]
	\column{0.4\textwidth}
	\frame{\includegraphics[width=\linewidth,height=0.75\paperheight]{example-image}}\\%
	\tiny{Image credit}\par
	\column{0.6\textwidth}
	\titlemessage{Vary slide layout to avoid monotony}
	\bigskip\bigskip
	But don't distract from the message \par
	\medskip Image on left when it makes sense to see image before title \par
	\begin{itemize}
		\item Introducing biology motivating the overall research question
		\item Introducing study system
		\item When image is only visual accesory
	\end{itemize}
\end{columns}
\end{frame}

%%
%% MODEL AND PREDICTIONS
%% 
\section{Model and predictions} 

% Progressive display
\begin{frame}{}
\begin{columns}[T]
	\column{0.5\textwidth}
	% {\usebeamerfont{title}\usebeamercolor[fg]{title}Use progressive display for clarity}\par
	\titlemessage{Use progressive display for clarity}
	\bigskip\bigskip
	\onslide<1->Don't confuse audience with too much at once \par
	\onslide<2->\medskip Use \texttt{onslide} to show slides bit by bit \par
	\onslide<3->\bigskip \alert{Highlight the specific research question your results will answer}
	\column{0.5\textwidth}
	\begin{centering}%
		\onslide<1->\frame{\includegraphics[width=\linewidth,height=0.35\paperheight]{example-image-a}}\par
		\onslide<2->\medskip\frame{\includegraphics[width=\linewidth,height=0.35\paperheight]{example-image-b}}\par
	\end{centering}
\end{columns}
\end{frame}

%% 
%% FIRST RESULTS SECTION
%% 
\section{First results section}

% Slide with single image
\begin{frame}{}
	\titlemessage{Slide with single image}
	\bigskip\bigskip
	\begin{columns}[T]
		\column{0.36\textwidth}
		Good for single-panel figure\par
		\medskip Describe setup if necessary \par
		\medskip State how data supports main conclusion
		\column{0.55\textwidth}
		\begin{centering}%
			\frame{\includegraphics[width=\linewidth]{example-image}}\par%
			% \includegraphics[width=\linewidth]{methods-exptl-design}\par%
		\end{centering}	
	\end{columns}
\end{frame}

% Wide image
\begin{frame}{}
	\titlemessage{Slide with wide image}
	\bigskip\medskip
	\begin{centering}%
		\frame{\includegraphics[width=\textwidth,height=0.4\paperheight]{example-image}}\par%
	\end{centering}
	\bigskip
	\begin{columns}[T]
		\column{0.41\textwidth}
		Good for multipart figures \par
		\column{0.5\textwidth}
		Use two columns of text if you need to
		\begin{itemize}
			\item Bottom of slide can be hard to see
		\end{itemize}
	\end{columns}
\end{frame}

%% 
%% SECOND RESULTS SECTION
%% 
\section{Second results section}

% Side by side images
\begin{frame}{}
	\titlemessage{Slide with side by side images}\bigskip\medskip
	\begin{columns}[T]
		\column{0.45\textwidth}
		\begin{overprint}
			\onslide<1>\raggedleft\frame{\includegraphics[width=\textwidth]{example-image}}\par%
			\onslide<2->\raggedleft\frame{\includegraphics[width=\textwidth]{example-image-a}}\par%
		\end{overprint}
		\onslide<1->{\bigskip Good for complex results \par}
		\onslide<2->{\smallskip Lets you explain axes \par}
		\column{0.45\textwidth}
		\begin{overprint}
			\onslide<-2>\raggedright\frame{\includegraphics[width=\textwidth]{example-image}}\par%
			\onslide<3->\raggedright\frame{\includegraphics[width=\textwidth]{example-image-b}}\par%
		\end{overprint}
		\onslide<3->\bigskip Then results one at a time \par
	\end{columns}
\end{frame}

% Blocks
%\begin{frame}{Block environments}
%\begin{columns}
%	\column{0.45\textwidth}
%	\begin{block}{Block title}
%		Block body text\par
%		\smallskip Useful for comparing things\par
%	\end{block}
%	\begin{block}{Another block title}
%		%More block body text\par
%		\begin{itemize}\setlength\itemsep{0pt}
%			\item First item
%			\item Second item
%		\end{itemize}
%	\end{block}
%	\begin{alertblock}{Alertblock title}
%		Alertblock body text
%	\end{alertblock}
%	\column{0.55\textwidth}
%	\begin{centering}%
%		\includegraphics[width=\linewidth]{intro-plasmid-transmission.pdf}%
%	\end{centering}	
%\end{columns}
%\end{frame}

%% 
%% DISCUSSION
%% 

% Further questions
\begin{frame}{}
\begin{columns}[T]
	\column{0.4\textwidth}
	\frame{\includegraphics[width=\linewidth,height=0.8\paperheight]{example-image}}\\%
	\tiny{Image credit}\par
	\column{0.6\textwidth}
	\titlemessage{Further questions}
	\bigskip\bigskip
	Good science opens up new questions \par
	\medskip Perhaps you got unexpected results that might be caused by something cool \par
	\medskip Want audience excited about your next projects \par
	\medskip They often have good ideas about what to explore \par
\end{columns}
\end{frame}

% Key findings
{\setbeamercolor{background canvas}{bg=matryoshkevGreyLight}
% \setbeamertemplate{background}{\includegraphics[width=\paperwidth,height=\paperheight]{example-image.jpg}}
\begin{frame}{}
	\begin{columns}
		\column{0.6\textwidth}
		\titlemessage{Key findings}\bigskip\bigskip
%		\begin{itemize}
%			\begin{normalsize}
%			\item Last slide should be the most important take-home messages
%			\item With a cool picture from your work
%			\item Then say "So, thanks for your attention. I'd be happy to take any questions."
%			\item Leave this slide up during questions
%			\end{normalsize}
%		\end{itemize}
		Last slide should be the most important take-home messages \par
		\medskip With a cool picture from your work \par
		\medskip Then say "So, thanks for your attention. I'd be happy to take any questions." \par
		\medskip Leave this slide up during questions \par
		\bigskip\bigskip
		\begin{scriptsize}
			Author \textit{et al.}~(YYYY) ``Published results'' \textit{Journal} VV:PPPP \par
			Author \textit{et al.}~(Preprint) ``Preprint results'' \textit{bioRxiv} \par
		\end{scriptsize}
		\column{0.4\textwidth}
		\strut
		% \frame{\includegraphics[width=\linewidth,height=0.85\paperheight]{example-image}}
	\end{columns}
\end{frame}
}

%%
%% EXTRA SLIDES
%% 

\section{Extra results}

\begin{frame}{}
	\titlemessage{Include less-important results as optional slides}
	\bigskip\medskip
	\begin{columns}[T]
		\column{0.4\textwidth}
		Often relevant to audience questions \par
		\medskip Or post-talk discussion \par
		\column{0.51\textwidth}
		\begin{centering}%
			\frame{\includegraphics[width=\textwidth]{example-image}}\par%
			% \includegraphics[width=\linewidth]{methods-exptl-design}\par%
		\end{centering}	
	\end{columns}
\end{frame}

%%
%% NAVIGATION
%%

% Use \section{} to create bookmarks for this slide
% Only include parts you'll likely want to go back to
\begin{frame}{}
	\tableofcontents
\end{frame}


\end{document}
